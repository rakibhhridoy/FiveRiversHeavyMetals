\section{Methodology}

\subsection{Study Area & Sample Collection}
The study was conducted on five major rivers around Dhaka, Bangladesh---the Buriganga, Shitalakshya, Turag, Dhaleshwari, and Balu---which are heavily influenced by industrial and domestic activities. Sampling was performed during two distinct seasons: the winter (November--February) and the rainy season (June--September), to capture seasonal variations in pollution. Sediment and water samples were collected from pre-determined stations near industrial outfalls and urban settlements.
Sediment samples were collected using a Van Veen grab sampler from a depth of 5--10 cm. Concurrently, surface water samples were collected using pre-cleaned polyethylene bottles. All samples were immediately placed in ice-cooled containers and transported to the laboratory to prevent physicochemical changes and preserve metal concentrations. 
\paragraph{Sediments:}Upon arrival, sediment samples were air-dried at room temperature, homogenized using an agate mortar and pestle, and sieved through a 2~mm mesh to remove coarse debris. The fine fraction was stored in polyethylene bags for subsequent analysis. 
\paragraph{Water:}Water samples were filtered using Whatman No.~42 filter paper to remove suspended particulates. On-site measurements were taken for key physicochemical parameters, including pH, turbidity, total dissolved solids (TDS), electrical conductivity (EC), and dissolved oxygen (DO), using portable multi-parameter meters.
For metal analysis, a 0.5~g portion of the homogenized sediment was digested using a mixture of concentrated nitric acid (\ce{HNO3}) and hydrochloric acid (\ce{HCl}) in a closed-vessel microwave digestion system. Filtered water samples (50~mL) were acidified with ultrapure \ce{HNO3} to a pH of $<$2.
The concentrations of heavy metals, including chromium (Cr), nickel (Ni), copper (Cu), arsenic (As), cadmium (Cd), and lead (Pb), were determined using Energy Dispersive X-Ray Fluorescence (EDXRF) spectrometry. The instrument was calibrated with certified reference materials (CRMs) for both sediments (e.g., MESS-4) and water to ensure accuracy. All measurements were performed in triplicate to ensure precision, and blank samples were analyzed to correct for any potential contamination.


\input{Algorithms2}


\subsection{Data Post Processing}
Following model training, a rigorous post-processing pipeline was implemented to evaluate predictive performance, interpret model behavior, and quantify the contribution of each data modality to the prediction of sediment risk index (RI). Model outputs (predicted RI values) were first compared against observed values using multiple complementary statistical indicators to capture both accuracy and robustness. The coefficient of determination (R2) was employed to quantify the proportion of variance in RI explained by the models, while the root mean square error (RMSE) and mean absolute error (MAE) provided measures of absolute prediction deviation and robustness against outliers. In addition, the symmetric mean absolute percentage error (SMAPE) was computed to standardize prediction errors relative to the magnitude of observed values, thereby ensuring comparability across different concentration ranges. Beyond predictive accuracy, feature attribution analyses were carried out to disentangle the contribution of each input modality--CNN-derived raster patches, MLP-based tabular attributes, and GNN-based spatial adjacency features--to the overall ensemble model. This was achieved through permutation feature importance, whereby the input values of each modality were randomly shuffled while holding the others constant, and the resulting decline in R 
2 was recorded as a measure of relative importance. This procedure enabled quantification of how strongly each data stream (spectral indices, interpolated heavy metal rasters, anthropogenic buffers, hydrological distances, or spatial autocorrelation matrices) contributed to predictive skill. CNN-based features, for example, captured fine-scale spatial variability in environmental indices such as NDVI, NDWI, and SAVI; MLP-based attributes reflected standardized geochemical and textural variables; and GNN-based inputs encoded spatial dependence across sites. Together, the permutation results were summarized as importance scores, ranked by predictive loss, and used to interpret which environmental and anthropogenic drivers were most influential during the rainy season. To complement these global insights, Local Interpretable Model-agnostic Explanations (LIME) were employed to interpret individual model predictions. LIME works by approximating the complex black-box model with a simpler, more interpretable surrogate model, similar to linear regression, in the local vicinity of a specific prediction. This was achieved by generating a new, perturbed dataset by randomly sampling instances around a given observation point, weighting these new instances by their proximity to the original observation, and training a simple, local model on this weighted dataset. By analyzing the coefficients of this local model, it is possible to identify which features most strongly influenced the predicted RI value for a specific sampling site. For instance, LIME could reveal that a high predicted RI at a particular location was predominantly driven by high concentrations of interpolated lead and its proximity to an industrial area, providing fine-grained, sample-level explanations critical for targeted remediation efforts. This post-processing framework, combining statistical evaluation, feature attribution, and spatial visualization, ensured that the predictive outputs of the ensemble models were not only quantitatively validated but also ecologically interpretable, facilitating a robust discussion on the sources and spatial dynamics of heavy metal contamination in riverine sediments.


