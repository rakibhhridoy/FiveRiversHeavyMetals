The Enrichment Factor (EF) is a widely used index for assessing the anthropogenic impact on sediment contamination. It quantifies the contribution of heavy metals to the sediment enrichment relative to background levels. The EF values were calculated for Cr, Ni, Cu, As, Cd, and Pb in the sediments from the Balu, Buriganga, Dhaleshwari, Shitalakshya, and Turag rivers during both the winter and rainy seasons. The EF values for each metal in the sediments are summarized in Table 2. The river system with the highest EF values was Shitalakshya, followed by Buriganga, Turag, Dhaleshwari, and Balu, indicating the degree of pollution in these rivers. The Shitalakshya river showed the highest enrichment in both seasons, particularly for Cd. In winter, the EF values for Cdranged from 28.03 to 48.96, and in the rainy season, they ranged from 28.66 to 38.63, indicating extremely high enrichment (EF ≥ 40) in winter and very high enrichment (20 ≤ EF < 40) in the rainy season. Similarly, Pb concentrations in Shitalakshya also exhibited significant enrichment, with EF values ranging from 4.47 to 12.20 in winter and 3.84 to 5.87 in the rainy season. These values indicate significant anthropogenic pollution, with industrial discharges and untreated waste likely contributing to the contamination (Hossain et al., 2021b). Following Shitalakshya, Buriganga exhibited the second-highest enrichment, especially for Cd and Pb. The EF values of Cd in Buriganga ranged from 18.23 to 21.01 in winter and 13.42 to 17.44 in the rainy season, indicating significant enrichment (5 ≤ EF < 20) in both seasons. The EF values of Pb in Buriganga ranged from 9.98 to 10.30 in winter and 5.63 to 5.86 in the rainy season, indicating moderate to strong enrichment. Turag exhibited moderate enrichment for Pb, with EF values ranging from 7.10 to 8.78 in winter and 3.32 to 5.50 in the rainy season. These findings suggest that Turag is moderately affected by industrial pollution and vehicular emissions, particularly in the winter season (Hossain et al., 2021b). The Dhaleshwari and Balu rivers showed moderate to significant enrichment for Cd and Pb, but with less severe pollution than Shitalakshya and Buriganga. In Dhaleshwari, the EF values of Cdranged from 3.60 to 15.52 in winter and 4.96 to 7.81 in the rainy season. Similarly, in Balu, Cdconcentrations showed moderate enrichment, with EF values ranging from 4.50 to 6.17 in the rainy season. Pb in Dhaleshwari and Balu exhibited lower EF values, ranging from 1.61 to 5.56 in winter and 1.19 to 2.94 in the rainy season, indicating minimal to moderate enrichment. In terms of Copper (Cu), the EF values were relatively lower, but still indicative of moderate enrichment in the Buriganga, Dhaleshwari, and Shitalakshya rivers. In Buriganga, the EF values of Cu ranged from 1.67 to 1.10 in winter and 1.10 to 1.60 in the rainy season, pointing to moderate contamination. Dhaleshwari and Shitalakshya also showed moderate enrichment for Cu, with EF values ranging from 1.00 to 2.00 in both seasons, suggesting a moderate anthropogenic impact. For metals like Chromium (Cr), Nickel (Ni), and Arsenic (As), the EF values were consistently low (EF < 2) in all rivers, indicating minimal to no anthropogenic enrichment. This suggests that these metals are either naturally occurring or have not been significantly impacted by human activities in these river sediments. 
Overall, the Shitalakshya and Buriganga rivers were found to be the most heavily polluted, with high EF values for Cd and Pb, indicating severe anthropogenic contamination, likely due to industrial discharges, vehicular emissions, and untreated sewage. The Turag, Dhaleshwari, and Balu rivers, while still contaminated, showed moderate enrichment for metals like Cd and Pb, suggesting that industrial and urban runoff have a moderate influence on these river systems. 
These results are consistent with findings from previous studies. Hossain et al. (2021b) reported moderate to severe enrichment of metals like Mn, Zn, Cu, Pb, Ni, and Cr in the sediments from Sitakundo coastal areas in Bangladesh, which align with the contamination levels observed in this study. Additionally, Tamim et al. (2016) found minimal enrichment of Cr, Zn, and other metals in the Buriganga River near the Hazaribagh area, indicating that, while certain areas of the river remain relatively unpolluted, others show significant industrial contamination. Based on the EF values, the rivers in Dhaka can be ordered in terms of pollution severity as: Shitalakshya > Buriganga > Turag > Dhaleshwari > Balu.