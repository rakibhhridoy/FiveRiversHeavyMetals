The Principal Component Analysis (PCA) conducted in this study reveals critical insights into the sources of heavy metal contamination in the urban rivers around Dhaka. The first three principal components (Dim.1, Dim.2, and Dim.3) collectively accounted for 88.6\% 
of the total variance, indicating that these components effectively capture the majority of the data's variability. Dim.1explained 47.4\%, 
Dim.2 explained 21.3\%, and Dim.3 accounted for the remaining variance, which is consistent with findings from other studies using PCA for environmental contamination assessments (Chowdhury et al., 2018). The PCA plot showed a clear separation of data points along these dimensions, confirming distinct patterns of contamination across the rivers. Further analysis through rotated components (RC1, RC2, RC3) revealed specific heavy metals that correlate strongly with each component. The first rotated component (RC1) showed strong correlations with Lead (Pb) (0.95) and Nickel (Ni) (0.91), suggesting that industrial emissions and vehicular sources are the primary contributors to the contamination in these rivers. This finding aligns with previous studies, which have reported that Pb and Ni are prevalent in industrial areas and are often associated with vehicular emissions (Bashar et al., 2019; Tariq et al., 2020). The second component (RC2), with significant positive loadings for Chromium (Cr) (0.95) and Copper (Cu) (0.72), suggests a mix of geological and anthropogenic sources, particularly from construction materials and industrial discharges (Rahman et al., 2020). The third component (RC3) was strongly dominated by Arsenic (As) (0.98), indicating that groundwater contamination or specific industrial processes (such as those in mining or textile industries) could be significant sources of Arsenic contamination, as previously noted in regions with industrial waste and agricultural runoff (Chowdhury et al., 2018). These results confirm that contamination in the rivers around Dhaka can be attributed to three major sources: industrial processes, vehicular emissions, and groundwater contamination. 