The Health Risk Assessment of sediments and water from the rivers surrounding Dhaka, Bangladesh, reveals severe contamination by key heavy metals such as Chromium (Cr), Nickel (Ni), Copper (Cu), Arsenic (As), Cadmium (Cd), and Lead (Pb), all of which exceed permissible limits set by environmental and health standards (USEPA, 2018). 
In sediments, Chromium, Nickel, and Arsenic present significant carcinogenic risks, with Chromium reaching 118.1 mg/kg in the winter, a level that poses a notable long-term cancer risk through ingestion or dermal contact (IARC, 2012). Lead, Copper, and Cadmium in sediments contribute primarily to non-carcinogenic risks, including neurological damage (especially in children), kidney and liver toxicity, as well as bone damage (ATSDR, 2007; EPA, 2020). These findings are in line with previous studies highlighting Lead as a significant neurotoxin, particularly in urban water systems (Chowdhury et al., 2018). The water quality assessment reveals that while the water concentrations of these metals, particularly Lead, Arsenic, and Copper, are generally lower than those in sediments, they still exceed the USEPA limits for safe drinking water. This indicates non-carcinogenic health risks such as gastrointestinal distress and liver toxicity from Copper, and neurological impairments from Lead (WHO, 2011; EPA, 2018). These metals, though somewhat diluted in the rainy season, remain at harmful levels, with Nickel and Arsenic still presenting significant risks, indicating persistent contamination from anthropogenic sources (Bashar et al., 2019). The study further suggests that winter months may exacerbate contamination levels due to reduced runoff, leading to more concentrated pollution from industrial discharge and domestic waste (Chowdhury et al., 2018). Despite the seasonal dilution during the rainy season, the findings indicate continuous pollution with metals like Nickel and Arsenic remaining at levels harmful to both human health and aquatic ecosystems. 

 