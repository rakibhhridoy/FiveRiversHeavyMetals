3.1. Heavy Metal Distribution 

The results from the sediments and water of urban rivers around Dhaka highlight substantial seasonal and spatial variability in the concentration of heavy metals. The data indicates that winter typically exhibits higher concentrations of pollutants than the rainy season, particularly due to lower water flow and higher industrial discharge during the drier months. In sediments, Chromium (Cr) concentrations averaged 106.58 mg/kg in winter and 92.69 mg/kg during the rainy season, with the highest concentration (118.1 mg/kg) recorded at Station S-12 in winter. These values are higher than those found in similar studies of urban rivers, which often see elevated levels of Cr due to industrial discharges (Chowdhury et al., 2018). Similarly, Nickel (Ni)levels ranged from 11.72 to 85.22 mg/kg, with mean concentrations of 51.44 mg/kg in winter and 42.13 mg/kg in the rainy season. Copper (Cu) displayed a significant seasonal difference, with an average of 117.55 mg/kg in winter, compared to 79.92 mg/kg during the rainy season. The highest concentration (261.1 mg/kg) was found at Station S-12 during winter, which can be attributed to industrial runoff, particularly from metal processing and battery manufacturing (Hossain et al., 2021). Arsenic (As) concentrations ranged from 7.49 to 28.37 mg/kg, with an average of 16.74 mg/kg in winter and 13.12 mg/kg in the rainy season. Cadmium (Cd) levels, though lower, showed distinct seasonal variation, averaging 1.34 mg/kg in winter and 0.83 mg/kg in the rainy season. These values are concerning because Cd is highly toxic even at low concentrations, leading to long-term environmental degradation (Rahman et al., 2020). Lead (Pb) concentrations were particularly high, with a mean of 78.74 mg/kg in winter, decreasing to 54.41 mg/kg during the rainy season. The highest concentration (297.4 mg/kg) was recorded at Station S-12 during winter, pointing to significant anthropogenic contamination, likely from automotive emissions and industrial effluents (Chowdhury et al., 2018). 
In river water, heavy metals also demonstrated significant seasonal variation. Chromium (Cr)concentrations averaged 0.22 mg/L in winter and 0.18 mg/L during the rainy season, with most values exceeding the WHO standard of 0.05 mg/L (World Health Organization, 2011). Nickel (Ni)concentrations were higher in winter (0.31 mg/L) than in the rainy season (0.25 mg/L), with peaks observed at Stations S-9 and S-10, indicative of localized pollution sources. Copper (Cu) levels averaged 1.98 mg/L in winter and 1.66 mg/L in the rainy season, with some stations exceeding the WHO permissible limit of 2.0 mg/L. This is particularly concerning for aquatic health, as Cu is known to be toxic to aquatic organisms (USEPA, 2018). Arsenic (As) concentrations ranged between 0.10 and 0.50 mg/L, with mean values of 0.33 mg/L in winter and 0.29 mg/L in the rainy season, both exceeding safety limits and posing significant health risks to populations relying on these water sources for drinking and agriculture (WHO, 2011). Cadmium (Cd) concentrations were relatively low, averaging 0.019 mg/L in winter and 0.014 mg/Lin rainy, but these levels still surpass permissible limits for drinking water, emphasizing the long-term dangers of Cd exposure (IARC, 2012). Lead (Pb) concentrations were 0.13 mg/L in winter and 0.10 mg/L in the rainy season, consistently exceeding WHO standards for safe drinking water. The highest Pb concentration in water (0.25 mg/L) was recorded at Station S-15 during winter, indicating significant pollution and potential risks to human health, including neurological damage and developmental delays in children (ATSDR, 2007). 
These findings clearly indicate that both sediment and water samples from the urban rivers of Dhaka frequently exceed permissible limits for several heavy metals, especially during the winter season. This has important implications for both ecological health and human well-being, as exposure to these metals can cause long-term environmental damage and increase the risk of chronic diseases. The high concentrations of Pb, Cu, and Cd in particular are of major concern, as they pose significant ecological risks, including toxicity to aquatic organisms and bioaccumulation through the food chain (Rahman et al., 2020). 