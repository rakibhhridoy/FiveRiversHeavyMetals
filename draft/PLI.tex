The limitations of single metal indices led to the development of multi-metal indices. The two most widely used such indices, developed by Hakinson (1980) and Nemerow (1991), include the modified degree of contamination (mCd) and the pollution index (PI). Brady et al. (2015) developed a modified pollution index (MPI) considering enrichment factor. The Pollution Index (PI), Modified Pollution Index (MPI), and Modified Degree of Contamination (mCd) values from this study indicate significant contamination in the urban rivers of Dhaka, with Shitalakshya and Buriganga showing the highest pollution levels. The PI values of 29.59 (winter) and 22.71 (rainy season) in Shitalakshya, and similar high values in Buriganga, confirm heavily polluted conditions (PI > 3) in both rivers. This is consistent with the findings of Chowdhury et al. (2018), who identified Buriganga as heavily polluted due to industrial effluents and urban runoff. Similarly, mCd values of 9.2 and 6.67 for Shitalakshya in winter and rainy seasons respectively indicate severe pollution (mCd > 8), while Buriganga recorded a mCd of 8.15 in winter. Ahsan et al. (2019) and Rahman et al. (2020) also reported high contamination in these rivers, with Shitalakshya and Buriganga being the most impacted by industrial waste. Turag, Dhaleshwari, and Balu rivers showed moderate pollution with mCd values ranging from 2 to 4, reflecting lower contamination compared to Shitalakshya and Buriganga. In the rainy season, these rivers showed some reduction in pollution levels, supporting the seasonal dilution effect observed in other studies (Bashar et al., 2019). The MPI values (>10) in Shitalakshya and Buriganga, and values between 5 and 10 in Turag and Dhaleshwari further highlight the varying levels of contamination, with Shitalakshya > Buriganga > Turag ≥ Dhaleshwari > Balu in both seasons. 