The criterion to evaluate the metal pollution in sediments is the Igeo that has been widely used since the late 1960s which is calculated by Eq (X) shown in Table X. In the present study, Igeo for the elements Cr, Ni, Cu, As, Cd and Pb is measured and presented in Table 3. The calculated Igeo values reveal that Cadmium (Cd) and Lead (Pb) exhibited the highest geo-accumulation indices in all five rivers, indicating significant contamination. In Shitalakshya and Buriganga, the Igeo values for Cd were notably high, indicating strong to extreme pollution in both seasons, with values of 4.74 (highest) in winter and 4.16 in rainy season. Similarly, the Igeo values for Cd in the Turag, Dhaleshwari, and Balu rivers ranged between 2.06 and 3.08 in winter, indicating moderate to strong pollution, with slightly lower values in the rainy season, ranging from 1.74 to 2.63, representing moderate pollution. The Igeo values for Pb in the Buriganga, Turag, and Shitalakshya rivers were found to be 3.21, 2.96, and 2.39 in winter, indicating strong pollution, and 2.47, 2.02, and 1.58 in the rainy season, suggesting moderate to strong pollution. In the Dhaleshwari and Balu rivers, Igeo values for Pbwere relatively lower, 1.34 and 1.05 in winter, indicating moderate pollution, and 0.45 and 0.17 in the rainy season, suggesting unpolluted to moderate pollution. Copper (Cu) exhibited moderate pollution in Buriganga, with Igeo values of 1.67 in winter and 1.10in the rainy season. Arsenic (As) and Copper (Cu) in Shitalakshya and Turag during the rainy season also showed relatively low values, indicating unpolluted to moderately polluted conditions. Metals like Chromium (Cr), Nickel (Ni), and Copper (Cu), across all five rivers in both seasons, showed negative Igeo values, indicating no significant pollution by these metals. This suggests that their concentrations in the sediments were within natural background levels. These findings are consistent with previous studies that used Igeo to assess sediment pollution. For example, Rakib et al. (2021a) found that the Igeo values for metals like Ti, Fe, Cu, Rb, Sr, Zr, Pb, and Zn in the sediments from marine coastal areas in Sitakundo, Bangladesh, were classified as class zero, indicating unpolluted sediments by these metals.  

 