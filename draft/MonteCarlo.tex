The Monte Carlo simulation for the Ecological Risk Index (RI), conducted over 10,000 iterations, provides a robust assessment of ecological risks associated with heavy metal contamination in the studied rivers. By applying both normal and lognormal distributions, the simulation effectively captures the range of potential ecological risks under different pollution scenarios. The normal distribution revealed moderate ecological risks, with a mean RI value of 250.6 and a maximum of 817.4, indicating a positive skew. This suggests that most of the study area experiences moderate ecological risks, likely stemming from chronic pollution sources such as industrial runoff and urban waste (USEPA, 2018). The 1st quartile (161.5) and median (235.2) values further support this conclusion, indicating that a significant portion of the area is subjected to moderate pollution levels. In contrast, the lognormal distribution, which better reflects the skewed nature of environmental data, yielded a mean of 254.78 and a maximum of 2205.86. These results underscore the potential for severe ecological risks in worst-case scenarios, such as industrial spills or large-scale contamination events, aligning with findings from Chowdhury et al. (2018), which highlighted the significant impact of industrial pollution in the region. Sensitivity analysis of the simulation revealed that Lead (Pb) posed the most significant ecological risk, contributing to 60.68\% of the total ecological risk in the rainy season. Other metals, such as Copper (Cu) (17.17\%) and Chromium (Cr) (9.04\%), 
also played crucial roles in shaping the ecological risk profile. In the winter season, Lead (Pb) continued to dominate, albeit with a slight reduction in its contribution to 60.68\%, while Cadmium (Cd) and Copper (Cu) also contributed substantially. This clearly indicates that Lead (Pb) and Copper (Cu) are the primary contributors to ecological risks in both seasons, reinforcing findings from earlier studies that emphasized the dominance of Pb and Cu in heavily polluted river systems (Rahman et al., 2020). These findings emphasize the urgent need for targeted pollution control measures, particularly in industrial areas, where Lead and Copper are likely being released into the rivers. The results also highlight the importance of continuous environmental monitoring to address both chronic and extreme pollution risks effectively. Early warning systems and improved industrial waste management strategies are essential to mitigate both ecological and health risks, ensuring long-term sustainability for the affected river systems and surrounding communities. 